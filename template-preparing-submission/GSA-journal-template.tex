\documentclass[9pt,twocolumn,twoside]{gsajnl}
% Use the documentclass option 'lineno' to view line numbers

\articletype{inv} % article type
% {inv} Investigation 
% {gs} Genomic Selection
% {goi} Genetics of Immunity 
% {gos} Genetics of Sex 
% {mp} Multiparental Populations
\usepackage{cite}
\usepackage{url}


\title{New insights of the RNA-seq data of \textit{The Cancer Genome Atlas} for Papillary Thyroid Carcinoma}

\author[$\ast$,1]{Luís Revilla}
\author[$\ast$]{Marina Reixachs}
\author[$\ast$]{Mauro Álvarez}
\author[$\ast$]{Inés Sentís}


\affil[$\ast$]{Universitat Pompeu Fabra}


\keywords{Cancer; Genomics; Thyroid; ...}

\runningtitle{GENETICS Journal Template on Overleaf} % For use in the footer 

\correspondingauthor{Corresponding Author}

\begin{abstract}
The abstract should be written for people who may not read the entire paper, so it must stand on its own. The impression it makes usually determines whether the reader will go on to read the article, so the abstract must be engaging, clear, and concise. In addition, the abstract may be the only part of the article that is indexed in databases, so it must accurately reflect the content of the article. A well-written abstract is the  most effective way to reach intended readers, leading to more robust search, retrieval, and usage of the article. 

Please see additional guidelines notes on preparing your abstract below.
\end{abstract}

\setboolean{displaycopyright}{true}

\begin{document}

\maketitle
\thispagestyle{firststyle}
\marginmark
\firstpagefootnote
\correspondingauthoraffiliation{Please insert the affiliation correspondence address and email for the corresponding author. The corresponding author should be marked with a `1' in the author list, as shown in the example.}
\vspace{-11pt}%

\lettrine[lines=2]{\color{color2}T}{}his \textit{Genetics} journal template is provided to help you write your work in the correct journal format. Instructions for use are provided below. 


\section*{Your Abstract}

In addition to the guidelines provided in the example abstract above, your abstract should:

\begin{itemize}
\item provide a synopsis of the entire article;
\item begin with the broad context of the study, followed by specific background for the study;
\item describe the purpose, methods and procedures, core findings and results, and conclusions of the study;
\item emphasize new or important aspects of the research;
\item engage the broad readership of GENETICS and be understandable to a diverse audience (avoid using jargon);
\item be a single paragraph of less than 250 words;
\item contain the full name of the organism studied;
\item NOT contain citations or abbreviations.
\end{itemize}

\section*{Introduction}

Thyroid cancer can develop from the different cells that form the follicles of the thyroid. This gland located at the base of the throat secretes hormones such as T3 and T4 that have their metabolic functionalities such as control of heart rate, blood pressure, body temperature and weight (see it on \href{http://cancergenome.nih.gov/cancersselected/thyroid}{ TCGA}).

From the four types of thyroid cancer (papillary, follicular, medullary, and anaplastic thyroid cancer) Papillary Thyroid Carcinoma (PTC) is the most common type of thyroid cancer \citep{Agrawal2014a}. It is more prevalent in women than men and its common diagnosis occurs between 25 and 65 years old (see it \href{http://www.cancer.org/cancer/thyroidcancer/detailedguide/thyroid-cancer-risk-factors}{here}).

Driver onco-mutations for this type of cancer appear as alterations of the MAPK signaling pathway and the PI3K-AKT pathway \citep{Kimura2003}. Both are implied in cell proliferation and survival and in human tumorigenesis. The overactivation of the MAPK pathway because of mutations such as the BRAFV600E mutation, leads to the development of papillary thyroid cancer (PTC) from follicular thyroid cells \citep{Xing2013a}. On the other hand, mutations that activate the PI3K-AKT pathway, such as mutations in RAS, PTEN and PIK3CA, mostly leads to development of follicular thyroid adenoma (FTA) and follicular thyroid cancer (FTC) also in follicular thyroid cells \citep{Xing2013a}.

Research from \href{http://cancergenome.nih.gov/cancersselected/thyroid}{ The Cancer Genome Atlas} focused on PTC and they found that a part from alterations in BRAF (specifically BRAFV600E) and RAS, there are other driver mutations such the ones in EIF1AX PPM1D, and CHEK2 genes that are main alterations for that type of thyroid cancer . They proposed a new classification of PTC subtypes based on their molecular findings. 

The aim of our project deviates from being a replicate of the TCGA project. Our academic purpose has been to study differentially expressed genes from tumor and normal samples but also taking into account whether there could be a gender effect on the tumorgenesis of PTC. To our knowledge, there is no clear reason why thyroid cancer is more prevalent in women than men. Therefore, it would be interesting to analyze whether there are differentially expressed genes within genders and respect to thyroid cancer. In addition, in case there are significant differences in gene expression which are the main pathways affected. 


\section*{Materials and Methods}

\subsection*{Data Availability}

For this project we used publicly available RNA-seq data from \href{http://cancergenome.nih.gov/cancersselected/thyroid}{TCGA} consisting on 513 tumor samples from which 59 had paired normal samples \citep{Agrawal2014a}. The RDS file with the data is provided.
From the available data we selected a subset of interest formed by all paired samples from "white" and "not hispanic or latino" population, which were the most abundant population groups in the samples (see Supplementary Materials). Additionally we selected samples whose tumoral type correspond to the most common cancer in thyroids: Papillary Thyroid Carcinoma (PTC). With this approach we want to avoid to include in the analysis, samples that introduce variability (mainly differential expression due to environmental or epigenetic factors) as much as possible.

\subsection*{Normalization}
All the analysis presented in this study has been done using the R statistical programming language \citep{R}. Data has been managed within \textit{SummarizedExperiment} class object container \citep{SummarizedExperiment} of the Bioconductor open-source and open-development software for R \citep{Gentleman2004}. Before any analysis, we have normalized the data following the edgeR pipeline for normalizing RNA-seq data \citep{Robinson2010}. Robinson's guide consist on first of all scaling the data into \textit{counts per million reads} (CPM) and then using the \textit{Trimmed Mean of M-values}(TMM) method implemented in the edgeR R/Bioconductor package \citep{Robinson2010a}. After that, we also have perfomed a quantile normalization on the RNA-seq count data using the \textit{cqn} R/Bioconductor package \citep{Hansen2012}. 

\subsection*{Batch effects assessment}
A part from normalization, we also have considered possible batch effects. 

\subsection*{Differential expression analysis}
In order to perform the differential expression analysis we used \textit{limma} R/Bioconductor package with a workflow based on what has been previously published \citep{limma}. We build the design matrix and calculated the observational-level weights. As we were working with paired data we estimated the correlation in repeated measurements before fitting the linear model. Finally we build and fitted contrast matrices related to gender and disease status (normal or tumor) and calculated moderated t-statistics. Moreover we used the \textit{sva} R/Bioconductor package to remove the effect of the surrogate variables \citep{sva}. 

\subsection*{Functional enrichment}
 

\section*{Results and Discussion}

The results and discussion should not be repetitive. The results section should give a factual presentation of the data and all tables and figures should be referenced; the discussion should not summarize the results but provide an interpretation of the results, and should clearly delineate between the findings of the particular study and the possible impact of those findings in a larger context. Authors are encouraged to cite recent work relevant to their interpretations. Present and discuss results only once, not in both the Results and Discussion sections. It is sometimes acceptable to combine results and discussion. The text should be as succinct as possible. Heed Strunk and White's dictum: "Omit needless words!"

\section*{Additional guidelines}

\subsection*{Numbers} In the text, write out numbers nine or less except as part of a date, a fraction or decimal, a percentage, or a unit of measurement. Use Arabic numbers for those larger than nine, except as the first word of a sentence; however, try to avoid starting a sentence with such a number.

\subsection*{Units} Use abbreviations of the customary units of measurement only when they are preceded by a number: "3 min" but "several minutes". Write "percent" as one word, except when used with a number: "several percent" but "75\%." To indicate temperature in centigrade, use ° (for example, 37°); include a letter after the degree symbol only when some other scale is intended (for example, 45°K).

\subsection*{Nomenclature and Italicization} Italicize names of organisms even when  when the species is not indicated.  Italicize the first three letters of the names of restriction enzyme cleavage sites, as in HindIII. Write the names of strains in roman except when incorporating specific genotypic designations. Italicize genotype names and symbols, including all components of alleles, but not when the name of a gene is the same as the name of an enzyme. Do not use "+" to indicate wild type. Carefully distinguish between genotype (italicized) and phenotype (not italicized) in both the writing and the symbolism.

\section*{In-text Citations}

Add citations using the \verb|\citep{}| command, for example \citep{neher2013genealogies} or for multiple citations, \citep{neher2013genealogies, rodelsperger2014characterization}

\section*{Examples of Article Components}
\label{sec:examples}

The sections below show examples of different header levels, which you can use in the primary sections of the manuscript (Results, Discussion, etc.) to organize your content.

\section*{First level section header}

Use this level to group two or more closely related headings in a long article.

\subsection*{Second level section header}

Second level section text.

\subsubsection*{Third level section header:}

Third level section text. These headings may be numbered, but only when the numbers must be cited in the text. 

\section*{Figures and Tables}

Figures and Tables should be labelled and referenced in the standard way using the \verb|\label{}| and \verb|\ref{}| commands.

\subsection*{Sample Figure}

Figure \ref{fig:spectrum} shows an example figure.

\begin{figure}[htbp]
\centering
\includegraphics[width=\linewidth]{example-figure}
\caption{Example figure from \url{10.1534/genetics.114.173807}. Please include your figures in the manuscript for the review process. You can upload figures to Overleaf via the Project menu. Upon acceptance, we'll ask for your figure files to be uploaded in any of the following formats: TIFF (.tiff), JPEG (.jpg), Microsoft PowerPoint (.ppt), EPS (.eps), or Adobe Illustrator (.ai).  Images should be a minimum of 300 dpi in resolution and 500 dpi minimum if line art images.  RGB, CMYK, and Grayscale are all acceptable. Halftones should be high contrast with sharp detail, because some loss of detail and contrast is inevitable in the production process. Figures should be 10-20 cm in width and 1-25 cm in height. Graph axes must be exactly perpendicular and all lines of equal density.
Label multiple figure parts with A, B, etc. in bolded type, and use Arrows and numbers to draw attention to areas you want to highlight. Legends should start with a brief title and should be a self-contained description of the content of the figure that provides enough detail to fully understand the data presented. All conventional symbols used to indicate figure data points are available for typesetting; unconventional symbols should not be used. Italicize all mathematical variables (both in the figure legend and figure) , genotypes, and additional symbols that are normally italicized.  
}%
\label{fig:spectrum}
\end{figure}

\subsection*{Sample Video}

Figure \ref{video:spectrum} shows how to include a video in your manuscript.

\begin{figure}[htbp]
\centering
\includegraphics[width=\linewidth]{example-figure}
\caption{Example movie (the figure file above is used as a placeholder for this example). \textit{GENETICS} supports video and movie files that can be linked from any portion of the article - including the abstract. Acceptable formats include .asf, avi, .wav, and all types of Windows Media files.   
}%
\label{video:spectrum}
\end{figure}


\subsection*{Sample Table}

Table \ref{tab:shape-functions} shows an example table. Avoid shading, color type, line drawings, graphics, or other illustrations within tables. Use tables for data only; present drawings, graphics, and illustrations as separate figures. Histograms should not be used to present data that can be captured easily in text or small tables, as they take up much more space.  

Tables numbers are given in Arabic numerals. Tables should not be numbered 1A, 1B, etc., but if necessary, interior parts of the table can be labeled A, B, etc. for easy reference in the text.  


\begin{table*}[htbp]
\centering
\caption{\bf Students and their grades}
\begin{tableminipage}{\textwidth}
\begin{tabularx}{\textwidth}{XXXX}
\hline
Student & Grade\footnote{This is an example of a footnote in a table. Lowercase, superscript italic letters (a, b, c, etc.) are used by default. You can also use *, **, and *** to indicate conventional levels of statistical significance, explained below the table.} & Rank & Notes \\
\hline
Alice & 82\% & 1 & Performed very well.\\
Bob & 65\% & 3 & Not up to his usual standard.\\
Charlie & 73\% & 2 & A good attempt.\\
\hline
\end{tabularx}
  \label{tab:shape-functions}
\end{tableminipage}
\end{table*}

\section*{Sample Equation}

Let $X_1, X_2, \ldots, X_n$ be a sequence of independent and identically distributed random variables with $\text{E}[X_i] = \mu$ and $\text{Var}[X_i] = \sigma^2 < \infty$, and let
\begin{equation}
S_n = \frac{X_1 + X_2 + \cdots + X_n}{n}
      = \frac{1}{n}\sum_{i}^{n} X_i
\label{eq:refname1}
\end{equation}
denote their mean. Then as $n$ approaches infinity, the random variables $\sqrt{n}(S_n - \mu)$ converge in distribution to a normal $\mathcal{N}(0, \sigma^2)$.

\bibliography{IEO}

\end{document}